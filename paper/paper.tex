\documentclass[12pt,halfline,a4paper,]{ouparticle}

% Packages I think are necessary for basic Rmarkdown functionality
\usepackage{hyperref}
\usepackage{graphicx}
\usepackage{listings}
\usepackage{color}
\usepackage{fancyvrb}
\usepackage{framed}

% For knitr::kable functionality
\usepackage{booktabs}
\usepackage{longtable}

%% To allow better options for figure placement
%\usepackage{float}

% Packages that are supposedly required by OUP sty file
\usepackage{amssymb, amsmath, geometry, amsfonts, verbatim, endnotes, setspace}

% For code highlighting I think
\DefineVerbatimEnvironment{Highlighting}{Verbatim}{commandchars=\\\{\}}
\definecolor{shadecolor}{RGB}{248,248,248}
\newenvironment{Shaded}{\begin{snugshade}}{\end{snugshade}}
\newcommand{\AlertTok}[1]{\textcolor[rgb]{0.94,0.16,0.16}{#1}}
\newcommand{\AnnotationTok}[1]{\textcolor[rgb]{0.56,0.35,0.01}{\textbf{\textit{#1}}}}
\newcommand{\AttributeTok}[1]{\textcolor[rgb]{0.77,0.63,0.00}{#1}}
\newcommand{\BaseNTok}[1]{\textcolor[rgb]{0.00,0.00,0.81}{#1}}
\newcommand{\BuiltInTok}[1]{#1}
\newcommand{\CharTok}[1]{\textcolor[rgb]{0.31,0.60,0.02}{#1}}
\newcommand{\CommentTok}[1]{\textcolor[rgb]{0.56,0.35,0.01}{\textit{#1}}}
\newcommand{\CommentVarTok}[1]{\textcolor[rgb]{0.56,0.35,0.01}{\textbf{\textit{#1}}}}
\newcommand{\ConstantTok}[1]{\textcolor[rgb]{0.00,0.00,0.00}{#1}}
\newcommand{\ControlFlowTok}[1]{\textcolor[rgb]{0.13,0.29,0.53}{\textbf{#1}}}
\newcommand{\DataTypeTok}[1]{\textcolor[rgb]{0.13,0.29,0.53}{#1}}
\newcommand{\DecValTok}[1]{\textcolor[rgb]{0.00,0.00,0.81}{#1}}
\newcommand{\DocumentationTok}[1]{\textcolor[rgb]{0.56,0.35,0.01}{\textbf{\textit{#1}}}}
\newcommand{\ErrorTok}[1]{\textcolor[rgb]{0.64,0.00,0.00}{\textbf{#1}}}
\newcommand{\ExtensionTok}[1]{#1}
\newcommand{\FloatTok}[1]{\textcolor[rgb]{0.00,0.00,0.81}{#1}}
\newcommand{\FunctionTok}[1]{\textcolor[rgb]{0.00,0.00,0.00}{#1}}
\newcommand{\ImportTok}[1]{#1}
\newcommand{\InformationTok}[1]{\textcolor[rgb]{0.56,0.35,0.01}{\textbf{\textit{#1}}}}
\newcommand{\KeywordTok}[1]{\textcolor[rgb]{0.13,0.29,0.53}{\textbf{#1}}}
\newcommand{\NormalTok}[1]{#1}
\newcommand{\OperatorTok}[1]{\textcolor[rgb]{0.81,0.36,0.00}{\textbf{#1}}}
\newcommand{\OtherTok}[1]{\textcolor[rgb]{0.56,0.35,0.01}{#1}}
\newcommand{\PreprocessorTok}[1]{\textcolor[rgb]{0.56,0.35,0.01}{\textit{#1}}}
\newcommand{\RegionMarkerTok}[1]{#1}
\newcommand{\SpecialCharTok}[1]{\textcolor[rgb]{0.00,0.00,0.00}{#1}}
\newcommand{\SpecialStringTok}[1]{\textcolor[rgb]{0.31,0.60,0.02}{#1}}
\newcommand{\StringTok}[1]{\textcolor[rgb]{0.31,0.60,0.02}{#1}}
\newcommand{\VariableTok}[1]{\textcolor[rgb]{0.00,0.00,0.00}{#1}}
\newcommand{\VerbatimStringTok}[1]{\textcolor[rgb]{0.31,0.60,0.02}{#1}}
\newcommand{\WarningTok}[1]{\textcolor[rgb]{0.56,0.35,0.01}{\textbf{\textit{#1}}}}

% For making Rmarkdown lists
\providecommand{\tightlist}{%
  \setlength{\itemsep}{0pt}\setlength{\parskip}{0pt}}

% Part for setting citation format package: natbib

% Part for setting citation format package: biblatex

% Part for indenting CSL refs
% Pandoc citation processing
% Pandoc header
\usepackage{booktabs}
\usepackage{longtable}
\usepackage{array}
\usepackage{multirow}
\usepackage{wrapfig}
\usepackage{float}
\usepackage{colortbl}
\usepackage{pdflscape}
\usepackage{tabu}
\usepackage{threeparttable}
\usepackage{threeparttablex}
\usepackage[normalem]{ulem}
\usepackage{makecell}
\usepackage{xcolor}

\begin{document}

\title{Financial Cycles in China}

\author{%
\name{Z. WANG}\address{Australian National University}\email{\href{mailto:zem.wang@anu.edu.au}{zem.wang@anu.edu.au}}
}

\abstract{This is the abstract.}

\date{\today}

\keywords{key; dictionary; word}

\maketitle



\hypertarget{introduction}{%
\section{Introduction}\label{introduction}}

The Global Financial Crisis has prompted a rethink of the role of
finance to the real economy. Though the view that financial factors
generate economic fluctuations has been along for centuries, the
mainstream neoclassical thinking is that ``money is a veil'' and
financial institutions are just intermediaries which alters no substance
in the real economy.

However, the onset of GFC demonstrated the great disruptions that
financial factors could bring about. It is clear that the relationship
between the ``Wall Street'' and the ``Main Street'' is not fully
captured by just interest rates -- as it is commonly assumed in
neoclassical literature.

After the GFC, growing efforts have been devoted to bridge the gap
between finance and real economy. On the theoretical side, notably,
Gertler and Kiyotaki (2011). Empirical literature have focused on
capturing the regularities of financial cycles and how it intereacts
with real business cycles (Claessens, 2011).

\hypertarget{data-and-methodology}{%
\section{Data and methodology}\label{data-and-methodology}}

\hypertarget{choice-of-variables}{%
\subsection{Choice of variables}\label{choice-of-variables}}

I choose three indicators to characterize the financial cycles: credit,
house price and credit to GDP ratio. Credit reflects the aggregate
claims created by the financial system, which is a natural choice to
represent financial cycles. Credit is also regarded by some as the
single most powerful indicator for financial crisis (Schularick \&
Taylor, 2012). In this paper, I use the Total Credit to the
Non-Financial Sector from the BIS credit statistics.\footnote{BIS total
  credit comprises financing from all sources, including domestic banks,
  other domestic financial corporations, non-financial corporations and
  non-residents. The financial instruments covered comprise currency and
  deposits, loans and debt securities. Debt securities include bonds and
  short-term papers; equities, insurance, pension funds are not
  included. In Chinese literature, the Aggregate Financing to the Real
  Economy (AFRE) indicator is more widely used. However, the AFRE
  statistics are only available yearly after 2002, and monthly after
  2014. The coverage of BIS total credit and AFRE largely overlapped.
  The correlation between BIS total credit and AFRE after 2014 is
  0.9944. To preserve consistency in data, BIS total credit is used
  throughout in this paper.} To measure how much credit is
``over-expanded'' relative to the size of the economy, credit to GDP
ratio is calculated and included as a separate indicator.

House price is another important indicator of financial cycles. House
properties, though being real estate, have many similar characteristics
as financial assets. As the most commonly accepted collateral for
credit, house prices capture the endogenous relationship between
entities' net worth, borrowing constraints, and credit creation. Strong
evidence suggests household leverage (mainly through mortgage loans)
influences business cycles through credit supply channels (Mian et al.,
2017). In this paper, I use the Average Price of Commercial Residential
Buildings from the National Bureau of Statistics as a measure of house
prices.\footnote{This only includes a portion of the total residential
  properties in China, indemnificatory housing and other types of
  public-built private-owned housing are not included in this
  statistics. But narrowing down to privately traded residential
  properties is a better choice to reflect market conditions. In Chinese
  literature, other indicators, such as average house price of 70 major
  cities, are more widely used. However, these indicators are only
  available for recent years. The indicator chosen in this paper traced
  the house prices back to 1998, the beginning of China's housing
  reform. }

The data are at quarterly frequency, ranging from 1999 to 2019. The data
starts from 1999, firstly due to the data availability; secondly,
considering China took a major reform in the housing market in July 1998
-- since then residential properties can be freely traded in the market
-- the house prices before 1999 have little value for the purpose of
this paper. The time series are seasonally adjusted whenever
necessary.\footnote{Except for GDP, I do not adjust the data for
  inflation. For the purpose of studying financial cycles, it is the
  nominal values that matter. } The original time series of the main
indicators are depicted in Figure \ref{fig:raw_data}.

\begin{figure}[h]

{\centering \includegraphics[width=1\linewidth]{paper_files/figure-latex/raw_data-1} 

}

\caption{Time series of main indicators (YoY)}\label{fig:raw_data}
\end{figure}

\hypertarget{cycle-extraction}{%
\subsection{Cycle extraction}\label{cycle-extraction}}

There are many methodologies to identify cycles in macroeconomic
variables. The classical method tracks absolute increase or decrease of
economic activities. For example, a recession is identified by two
consecutive quarters of declines of GDP. However, this ``classical''
business cycle definition is not suitable for fast-growing economies, in
which macroeconomic variables are constantly in upward movement. For
fast-growing economies, the ``growth cycle methodology'' is more
appropriate, which identifies cycles as fluctuations around a trend. In
this paper, I apply the Hodrick-Prescott (HP) filter to extract the
cyclical components from the time series.\footnote{The lambda value is
  set to 1600 for quarterly data as suggested by Hodrick \& Prescott
  (1997). There are other methods used for cycle extraction in the
  literature. For example, Ma et al. (2016) applied HP filter to YoY
  changes of variables. This method is problematic, because HP filter is
  designed to obtain trends by smoothing out short-term fluctuations.
  Performing year-to-year difference by default removes the trend.
  Another popular approach is to apply Bank-Pass filters such as
  Christiano-Fitzgerald (CF) filter. This is the approach adopted by
  Drehmann et al. (2012) and Zhu \& Huang (2018). However, CF filter
  pre-imposes the length of the cycle to be within a certain band.
  According to Drehmann et al. (2012), the short-term cycles are between
  1 and 8 years, and mid-term cycles are between 8 to 30 years. Since
  China's data is available for merely 20 years, applying the CF filter
  is unnecessary and bears the risk of losing too much information.}
Log-transformation is applied to variables that grow proportionately to
their sizes such as GDP, credit and house prices. The magnitude of the
cycles are standardized by their starndard deviation.

Harding \& Pagan (2002)'s (medium-term) turning-point algorithm is then
applied to identify peaks and troughs of the cycles. Specifically, the
algorithm involves two steps: (1) identifying local maxima and minima
over a window of 9 quarters; (2) imposing censoring rules that the
minimum length of each phase of upturn or downturn is 2 quarters;
Further, I impose a third restriction: (3) peaks and troughs must appear
alternately, that is there must be one and only one peaks (troughs)
between two troughs (peaks). Figure \ref{fig:cyl_tp} plots the HP
filtered cycles of the variables together with the identified turning
points.

\begin{figure}[h]

{\centering \includegraphics[width=1\linewidth]{paper_files/figure-latex/cyl_tp-1} 

}

\caption{Cyclical components and turning points}\label{fig:cyl_tp}
\end{figure}

\hypertarget{business-and-financial-cycles-basic-features}{%
\section{Business and financial cycles: basic
features}\label{business-and-financial-cycles-basic-features}}

\hypertarget{duration-and-amplitude}{%
\subsection{Duration and amplitude}\label{duration-and-amplitude}}

\begin{table}

\caption{\label{tab:dur_amp}Basic cyclical features of main variables}
\centering
\begin{tabular}[t]{llrrr}
\toprule
Indicator & Variable & Downturn & Upturn & Full Cycle\\
\midrule
 & Duration (Q) & 7.50 & 4.86 & 12.36\\
\cmidrule{2-5}
\multirow{-2}{*}{\raggedright\arraybackslash CPI} & Amplitude (pp.) & -1.04 & 1.15 & \\
\cmidrule{1-5}
 & Duration (Q) & 7.29 & 5.17 & 12.45\\
\cmidrule{2-5}
\multirow{-2}{*}{\raggedright\arraybackslash Credit} & Amplitude (\%) & -1.06 & 0.68 & \\
\cmidrule{1-5}
 & Duration (Q) & 11.75 & 7.40 & 19.15\\
\cmidrule{2-5}
\multirow{-2}{*}{\raggedright\arraybackslash Credit/GDP} & Amplitude (pp.) & -1.54 & 1.60 & \\
\cmidrule{1-5}
 & Duration (Q) & 9.80 & 8.25 & 18.05\\
\cmidrule{2-5}
\multirow{-2}{*}{\raggedright\arraybackslash GDP} & Amplitude (\%) & -1.55 & 1.30 & \\
\cmidrule{1-5}
 & Duration (Q) & 5.29 & 5.12 & 10.41\\
\cmidrule{2-5}
\multirow{-2}{*}{\raggedright\arraybackslash House} & Amplitude (\%) & -1.39 & 1.55 & \\
\cmidrule{1-5}
 & Duration (Q) & 6.50 & 6.50 & 13.00\\
\cmidrule{2-5}
\multirow{-2}{*}{\raggedright\arraybackslash R007} & Amplitude (pp.) & -1.23 & 1.94 & \\
\bottomrule
\end{tabular}
\end{table}

Based on the identified turning points in Section
\ref{cycle-extraction}, I calculated the duration and amplitude of the
cycles for each series (Table \ref{tab:dur_amp}). The duration of an
upturn (downturn) is defined as the quarters from a trough (peak) to a
peak (trough). The amplitude of an upturn (downturn) is defined as the
largest positive (negative) deviation from the trend.

Table \ref{tab:dur_amp} shows (1) Most variables have full cycles of
10-13 quarters, while GDP and Credit/GDP demonstrate considerable longer
cycles that lasts 18-19 quarters.\footnote{Spectrum analysis suggests
  GDP series has a 22-quarter cycle, CPI a 15-quarter cycle, property
  price a 22-quarter cycle and a 12-quarter cycle, credit a 18-quarter
  cycle and a 12-quarter cycle.} (2) Most variables exhibit longer
quarters of downturn than upturn. An upturn takes an average of 5-8
quarters, while a downturn generally takes 6-11 quarters. (3) For GDP
and credit, the downturns are more severe than the upturns (more
pronounced amplitude). While for houses, price levels, and interest
rates, the amplitude of troughs are relatively mild compared to
peaks.\footnote{The numbers here demonstrate drastically different
  characteristics from financial or business cycles concluded from
  developed economies (Borio, 2014; Claessens et al., 2012). In the
  later case, financial variables generally have longer cycles than real
  variables, and upturns are typically more pronounced in magnitude than
  downturns. The difference is partly due to the different methodology
  adopted in this paper. The ``growth cycle method'' is more sensitive
  to downturns than the ``classical method''. Institutional factors also
  affect the behavior of macroeconomic variables. For example, strong
  counter-cyclical fiscal policy is usually adopted in China during
  downturns. These factors make real GDP more stable than financial
  variables. The results here are consistent with conclusions from
  Chinese literature (Yi \& Zhang, 2016; Zhu \& Huang, 2018). }

\hypertarget{correlation-and-concordance}{%
\subsection{Correlation and
concordance}\label{correlation-and-concordance}}

Figure \ref{fig:corr} shows the correlation between the series. GDP and
CPI have a strong positive correlation, and both of them are positively
correlated with the interest rate. Credit is negatively correlated with
the interest rate, and hence negatively correlated with GDP and CPI.
House prices do not exhibit strong correlations with any other series.

\begin{figure}[h]

{\centering \includegraphics[width=1\linewidth]{paper_files/figure-latex/corr-1} 

}

\caption{Correlation between series}\label{fig:corr}
\end{figure}

To further shed some light on the concordance of the cycles of different
series, I calculated the concordance score between the variables as did
by Drehmann et al. (2012). Concordance measures the percentage of time
for which two series are in the same phase (upturn or downturn).
Specifically, the concordance scroe is defined as

\[ \rho_{XY} = \frac{1}{T} \sum_{t=1}^{T} [\rho_t^{X}\rho_t^{Y} + (1-\rho_t^{X})(1-\rho_t^{Y})]\]

where
\(\rho_t^{X} = \begin{cases}0, \text{if X is in downturn} \\1, \text{if X is in upturn} \end{cases}\),
\(\rho_t^{Y} = \begin{cases}0, \text{if Y is in downturn} \\1, \text{if Y is in upturn} \end{cases}\).

\begin{table}

\caption{\label{tab:cord}Concordance between variables}
\centering
\begin{tabular}[t]{lrrrrrr}
\toprule
Variables & CPI & Credit & Credit/GDP & GDP & House & R007\\
\midrule
CPI & 1.00 & 0.39 & 0.37 & 0.68 & 0.58 & 0.54\\
Credit & 0.39 & 1.00 & 0.62 & 0.55 & 0.48 & 0.64\\
Credit/GDP & 0.37 & 0.62 & 1.00 & 0.24 & 0.38 & 0.33\\
GDP & 0.68 & 0.55 & 0.24 & 1.00 & 0.57 & 0.67\\
House & 0.58 & 0.48 & 0.38 & 0.57 & 1.00 & 0.62\\
\addlinespace
R007 & 0.54 & 0.64 & 0.33 & 0.67 & 0.62 & 1.00\\
\bottomrule
\end{tabular}
\end{table}

The concordance scores (Table \ref{tab:cord}) shows that GDP and CPI
have 68\% of their cycles overlapped. Interest rate cycles overlap 67\%
of GDP cycles, 64\% of credit cycles, and 62\% of house price cycles.
Credit cycles have relatively low concordance with real variables, 55\%
with GDP, and 39\% with CPI. House prices have almost half of the cycles
overlapped with all other variables, though the concordance with
interest rate is slightly higher up to 62\%.

\hypertarget{dynamic-relations}{%
\section{Dynamic relations}\label{dynamic-relations}}

This section employs a VAR model to elicit the inter-relationship
between the variables of interest. The reduced-form of the model is as
following

\begin{equation}
y_t = c + \sum_{j=1}^{p} \Phi_j y_{t-j} + u_t 
\label{eq:var-model}
\end{equation}

where

\begin{equation}
y_t = [ 
  \text{GDP}_t, 
  \text{CPI}_t, 
  \text{IR}_t, 
  \text{Credit}_t, 
  \text{Credit/GDP}_t, 
  \text{House}_t 
  ] 
\label{eq:vars}
\end{equation}

All the variables included here are stationary (see Annex
\ref{annex-unit-root} for unit root test). Considering the limited size
of the data, I estimate the model with Bayesian method to avoid the
``curse of dimensionality''. I adopt the Minnesota prior and set the
prior AR(1) coefficient to zero.\footnote{Other parameters of the
  Minnesota prior are left with standard settings: \(\lambda_1=0.2\)
  (overall tightness), \(\lambda_2=1\) (relative cross-variable weight),
  \(\lambda_3=1\) (lag decay).} To identify the structural shocks,
Cholesky decomposition is applied to the residuals with regard to the
ordering of variables as in Equation (\ref{eq:vars}).\footnote{This
  ordering is justified as follows: the ordering of output, price level
  and interest rate follows directly from standard IS/LM models, in
  which output responses to price level only with lags, and central
  banks set interest rate in response to both output and price level. It
  is reasonable to expect credit follows interest rate, not the other
  way around. And credit fuels the purchasing of real estate. However,
  the results are robust to different ordering the variables. Annex
  \ref{annex-ordering} tests the sensitivity of the impulse responses to
  different ordering.}

The impulse responses of all variables are provided in Annex
\ref{annex-impulse}. I focus here on response of real variables to
proxies of financial cycles. Figure \ref{fig:imp_gdp} shows the
(accumulated) impulse response of GDP to innovations in credit, credit
to GDP ratio, and housing price. It shows that an unexpected increase in
all three variables leads to increase of real output, with housing price
the most prominent. The up trend peaks after roughly 10 quarters, then
the trend starts to reverse. This suggests a financial expansion
presages an immediate bump up in real output, and then a subsequent
decline.

\begin{figure}[h]

{\centering \includegraphics[width=0.49\linewidth]{../graph/imp_gdp} \includegraphics[width=0.49\linewidth]{../graph/imp_gdp_acc} 

}

\caption{Impulse response of GDP to innovations}\label{fig:imp_gdp}
\end{figure}

Figure \ref{fig:imp_cpi} shows the response of price level, measured by
CPI, to an unexpected upward movement of financial cycle. Price level
responds positively to all three innovations, but price is much more
sensitive to credit shocks than output. The up-moving trend of the price
level peaks in 9 quarters after a positive credit shock. The rising
house price has more enduring impact on price level -- the upward
movement lasts for 12 quarters before its reversal.

\begin{figure}[h]

{\centering \includegraphics[width=0.49\linewidth]{../graph/imp_cpi} \includegraphics[width=0.49\linewidth]{../graph/imp_cpi_acc} 

}

\caption{Impulse response of CPI to innovations}\label{fig:imp_cpi}
\end{figure}

Why the business cycle reverses its trend in two and a half years after
a financial expansion? The response of interest rate to financial cycle
might shed some light. As shown in Figure \ref{fig:imp_ir}, upward
movement of credit or housing price elicits rise in interest rate. The
response of interest rate is especially aggressive to rising housing
price. The rising interest rate tightens financial conditions and curbs
real economic activities. This possibly explains the reversal of the
business cycle after a short-run expansion.

\begin{figure}[h]

{\centering \includegraphics[width=0.49\linewidth]{../graph/imp_ir} \includegraphics[width=0.49\linewidth]{../graph/imp_ir_acc} 

}

\caption{Impulse response of interest rate to innovations}\label{fig:imp_ir}
\end{figure}

Based on the orthogonalised shocks, one can calculate the contributions
of each shocks to the development of economic process. This may help
understand the role of financial factors in shaping business cycles.
Figure \ref{fig:hd_gdp} decomposes the historical cycles of GDP and CPI
into different shocks. A full list of historical decomposition can be
found in Annex \ref{annex-hd-full}.

\begin{figure}[h]

{\centering \includegraphics[width=0.49\linewidth]{../graph/hd_gdp} \includegraphics[width=0.49\linewidth]{../graph/hd_cpi} 

}

\caption{Historical decomposition}\label{fig:hd_gdp}
\end{figure}

As shown in Figure \ref{fig:hd_gdp}, real GDP cycles are mostly
contributed by shocks to itself and shocks to CPI. Credit shocks played
an important role during before and during the GFC. The importance of
credit gradually moderated after the GFC. An exception is the year of
2012, in which credit shocks again played a prominent role. This was a
period when China's financial sector undertook ``innovations'' to
circumvent regulations.

House price historically played a prominent role in affecting the
business cycle. A notable feature is that house price tends to play a
supportive role when China's GDP swings below the growth trend. This can
be explained by that China's authority typically relax the policies that
restrict home purchase to counter economic downturn pressure. For
example, during the economic downturn in 2014, most cities remove the
limits on the number of homes one can buy. The importance of house price
attenuated since 2016, as the authority adopted the ``long-term
mechanism'' for real estate regulation.

\hypertarget{discussion}{%
\section{Discussion}\label{discussion}}

The last section in the document will be used as the section title for
the bibliography.

\hypertarget{reference}{%
\section{Reference}\label{reference}}

\hypertarget{refs}{}
\leavevmode\hypertarget{ref-Borio2014}{}%
Borio, C. (2014). The financial cycle and macroeconomics: What have we
learnt? \emph{Journal of Banking \& Finance}, \emph{45}, 182--198.
\url{https://doi.org/10.1016/j.jbankfin.2013.07.031}

\leavevmode\hypertarget{ref-Claessens2012}{}%
Claessens, S., Kose, M. A., \& Terrones, M. E. (2012). How do business
and financial cycles interact? \emph{Journal of International
Economics}, \emph{87}(1), 178--190.
\url{https://doi.org/10.1016/j.jbankfin.2013.07.031}

\leavevmode\hypertarget{ref-Drehmann2012}{}%
Drehmann, M., Borio, C. E., \& Tsatsaronis, K. (2012). Characterising
the financial cycle: Don't lose sight of the medium term! \emph{BIS
Working Paper}. \url{https://ssrn.com/abstract=2084835}

\leavevmode\hypertarget{ref-Harding2002}{}%
Harding, D., \& Pagan, A. (2002). Dissecting the cycle: A methodological
investigation. \emph{Journal of Monetary Economics}, \emph{49}(2),
365--381. \url{https://doi.org/10.1016/S0304-3932(01)00108-8}

\leavevmode\hypertarget{ref-Hodrick1997}{}%
Hodrick, R. J., \& Prescott, E. C. (1997). Postwar us business cycles:
An empirical investigation. \emph{Journal of Money, Credit, and
Banking}, 1--16. \url{https://doi.org/10.2307/2953682}

\leavevmode\hypertarget{ref-Mayong2016}{}%
Ma, Y., Feng, X., \& Tian, T. (2016). Financial cycle and economic
cycle: Evidence from china. \emph{Studies of International Finance
(Chinese)}, \emph{10}, 3--14.
\url{https://doi.org/10.16475/j.cnki.1006-1029.2016.10.001}

\leavevmode\hypertarget{ref-Mian2017}{}%
Mian, A., Sufi, A., \& Verner, E. (2017). Household debt and business
cycles worldwide. \emph{The Quarterly Journal of Economics},
\emph{132}(4), 1755--1817. \url{https://doi.org/10.1093/qje/qjx017}

\leavevmode\hypertarget{ref-Schularick2012}{}%
Schularick, M., \& Taylor, A. M. (2012). Credit booms gone bust:
Monetary policy, leverage cycles, and financial crises, 1870-2008.
\emph{American Economic Review}, \emph{102}(2), 1029--1061.
\url{https://doi.org/10.1257/aer.102.2.1029}

\leavevmode\hypertarget{ref-Yinan2016}{}%
Yi, N., \& Zhang, B. (2016). Measuring china's financial cycle.
\emph{Studies of International Finance (Chinese)}, \emph{06}, 13--23.
\url{https://doi.org/10.16475/j.cnki.1006-1029.2016.06.002}

\leavevmode\hypertarget{ref-Taihui2018}{}%
Zhu, T., \& Huang, H. (2018). China's financial cycle: Indicators,
methods and empirical research. \emph{Journal of Financial Research
(Chinese)}, \emph{462}, 55--71.
\url{http://www.jryj.org.cn/CN/abstract/article_550.shtml}

\hypertarget{appendix}{%
\section{Appendix}\label{appendix}}

\hypertarget{annex-unit-root}{%
\subsection*{Unit root test for the variables}\label{annex-unit-root}}
\addcontentsline{toc}{subsection}{Unit root test for the variables}

\begin{center}\includegraphics[width=0.5\linewidth]{../graph/ar_root} \end{center}

\hypertarget{annex-residuals}{%
\subsection*{Residual diagnosis of VAR model}\label{annex-residuals}}
\addcontentsline{toc}{subsection}{Residual diagnosis of VAR model}

\begin{center}\includegraphics[width=1\linewidth]{../graph/var_resid} \end{center}

\hypertarget{annex-impulse}{%
\subsection*{Impulse responses of all variables}\label{annex-impulse}}
\addcontentsline{toc}{subsection}{Impulse responses of all variables}

\begin{center}\includegraphics[width=1\linewidth]{../graph/imp_full} \end{center}

\begin{center}\includegraphics[width=1\linewidth]{../graph/imp_full_acc} \end{center}

\hypertarget{annex-hd-full}{%
\subsection*{Historical decomposition of all
variables}\label{annex-hd-full}}
\addcontentsline{toc}{subsection}{Historical decomposition of all
variables}

\begin{center}\includegraphics[width=1\linewidth]{../graph/hd_full} \end{center}

\hypertarget{annex-ordering}{%
\subsection*{Robustness to different ordering of
variables}\label{annex-ordering}}
\addcontentsline{toc}{subsection}{Robustness to different ordering of
variables}

\begin{center}\includegraphics[width=1\linewidth]{../graph/imp_ordering} \end{center}






\end{document}
